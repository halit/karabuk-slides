\documentclass[10pt, compress]{beamer}

\usetheme{m}

\usepackage{booktabs}
\usepackage[scale=2]{ccicons}
\usepackage{minted}
\usepackage[utf8]{inputenc}
\usepackage{wrapfig}

\usemintedstyle{trac}

\title{Python}
\subtitle{}
\date{27 Şubat 2015}
\author{Halit Alptekin \href{mailto:info@halitalptekin.com}{( info@halitalptekin.com} )}
\institute{Karabük}

\begin{document}

\maketitle

\begin{frame}[fragile]
    \frametitle{\$ whoami}
    \begin{description}
        \item[EDUC] Bilgisayar mühendisliği
        \item[WORK] Bilgi güvenliği
        \item[IDEA] Özgür yazılım ve açık kaynak tutkunu
        \item[MEMB] TMD, LKD, Octosec
        \item[LIFE] GO, matematik, kriptoloji
        \item[HOBB] A sınıfı amatör telsizci, elektronik, robotik
    \end{description}
\end{frame}

\section{Hacker}

\begin{frame}[fragile]
    \frametitle{Kimdir?}
    \begin{itemize}[<+- | alert@+>]
        \item Sorunlar için yaratıcı çözümler üretir
        \item Yaş, cinsiyet, ırk, dil, din ayrımı yapmaz
        \item Ürettiği yaratıcı çözümlerin, güzelliğine göre başarı kazanır
        \item Çözümlerin yanında sistemler üzerinde açık bulur
        \item MIT kökenli
    \end{itemize}  
\end{frame}

\plain{Gerçek Hacker?}{\vspace{-4em}\begin{center}\includegraphics[width=\textwidth, height=3.2in]{babanne.jpeg}\end{center}}

\section{Özgürlük}

\begin{frame}[fragile]
    \frametitle{Özgür Yazılım?}
    \begin{itemize}[<+- | alert@+>]
        \item Siyasi hareket
        \item Çalıştırma, kopyalama, dağıtma, inceleme, değiştirme ve geliştirme gibi özgürlükler ile ilgilenir
        \item Amaç yazılımları özgür kılmaktır
        \item Çeşitli lisanlar ile sağlanır(GPL, BSD vb)
    \end{itemize}  
\end{frame}

\begin{frame}[fragile]
    \frametitle{Açık Kaynak Yazılım?}
    \begin{itemize}[<+- | alert@+>]
        \item Sadece yazılımın kaynak kodlarının erişilebilir olduğunu vurgular
        \item Yazılım geliştirme metodolojisidir
        \item Özgür olmadan da açık kaynaklı olunabilir ancak tersi mümkün değildir
        \item Kodları açmak, yazılımı özgür bırakmak için yeterli değildir
    \end{itemize}  
\end{frame}


\section{Python}

\begin{frame}[fragile]
  \frametitle{Nedir?}
  \begin{center}
  Python, turing-complete bir programlama dilidir.
  \includegraphics[width=0.5\textwidth]{py.png}
  \end{center}
\end{frame}


\begin{frame}[fragile]
    \frametitle{Hikayesi?}
    \begin{itemize}[<+- | alert@+>]
        \item 1990 yılında Guido van Rossum tarafından geliştirilmeye başlanmıştır
        \item İsmini Monty Python isimli ingiliz gösteri grubundan almıştır
        \item Özellikle Hacker kültürünün Perl'den sonra sahiplenmesiyle ayrıcalıklı bir duruma gelmiştir
        \item Geliştirilmesi, Python Yazılım Vakfı çevresinde toplanan geliştiriciler ile sağlanmaktadır
        \item Ocak 1994 tarihinde 1.0 sürümü yayınlanmıştır
    \end{itemize}  
\end{frame}

\begin{frame}[fragile]
    \frametitle{Özellikleri?}
    \begin{itemize}[<+- | alert@+>]
        \item Genel amaçlı
        \item Yüksek seviyeli
        \item Çoklu paradigmalı
        \item Dinamik tip sistemi
    \end{itemize}  
\end{frame}

\plain{Python?}{\vspace{-3em}\begin{center}\includegraphics[width=\textwidth, height=3.2in]{python.png}\end{center}}

\begin{frame}[fragile]
    \frametitle{Artıları?}
    \begin{itemize}[<+- | alert@+>]
        \item Kod okunabilirliği
        \item Kolay öğrenme
        \item Sade ve basit sözdizimi
        \item Taşınabilir uygulamalar
        \item İçe bakış ve dökümantasyon
        \item Çok geniş kütüphane desteği
        \item Hızli geliştirme imkanı
        \item Özgür yazılım, açık kaynak kodlu
    \end{itemize}  
\end{frame}


\begin{frame}[fragile]
    \frametitle{Eksileri?}
    \begin{itemize}[<+- | alert@+>]
        \item Derlenmiyor, yorumlanıyor
        \item Mobil
        \item Çok kolay gözükmesi
    \end{itemize}  
\end{frame}

\plain{Python?}{\vspace{-4em}\begin{center}\includegraphics[width=\textwidth, height=3.2in]{pyjava.jpg}\end{center}}

\begin{frame}[fragile]
    \frametitle{Neler yapabilirim?}
    \begin{itemize}[<+- | alert@+>]
        \item Web sitesi
        \item Oyun
        \item Günlük araçlar
        \item Mobil uygulama
        \item Bilimsel çalışma\ldots 
    \end{itemize}  
\end{frame}

\begin{frame}[fragile]
    \frametitle{Kimler kullanıyor?}
    \begin{itemize}[<+- | alert@+>]
        \item Google, Yahoo, NASA
        \item Dropbox, Disqus, Mozilla
        \item Friendfeed, Reddit, Eventbrite
        \item Walt Disney, Battlefield 2, Civilization 4
        \item Nokia, IBM, CIA \ldots  
    \end{itemize}  
\end{frame}

\section{Nasıl?}

\begin{frame}[fragile]
\frametitle{Pip}
    Python'un kendi paket yöneticisi:
      \begin{minted}[fontsize=\small]{bash}
        apt-get install python-pip
        yum install python-pip
        pacman -S python2-pip (python-pip)
      \end{minted}    
    Temel komutları:
    \begin{minted}[fontsize=\small]{bash}
        pip install <paket_adi>
        pip search <paket_adi>
        pip uninstall <paket_adi>
        pip install -r requirements.txt
    \end{minted}       
\end{frame}

\begin{frame}[fragile]
\frametitle{Virtualenv}
    Python bağımlılıkları için izole bir ortam oluşturuyor. Yüklediğiniz paketler sistemin genelini etkilemiyor. Paketlerin farklı sürümleri ile çalışma imkanı sağlıyor.
    \begin{minted}[fontsize=\small]{bash}
        pip install virtualenv
    \end{minted}  
    Temel komutlar:
    \begin{minted}[fontsize=\small]{bash}
        cd <klasor_adi>
        virtualenv <sanal_ortam_adi>
        virtualenv -p /usr/bin/python2.7 <sanal_ortam_adi>
        source <sanal_ortam_adi>/bin/activate
        deactivate
    \end{minted}    
    Birleşmenin tam zamanı!
    \begin{minted}[fontsize=\small]{bash}
        pip freeze > requirements.txt
        pip install -r requirements.txt
    \end{minted}       
\end{frame}

\plain{IPython}{\vspace{-3em}\begin{center}\includegraphics[width=\textwidth, height=3.2in]{ip.png}\end{center}}


\begin{frame}[fragile]
    \frametitle{Pyserial}
    \begin{center}
        Arduino ile haberleşmek istiyorum?
    
      \begin{minted}[fontsize=\small]{python}
        import serial
        
        ser = serial.Serial('/dev/tty.usbserial', 9600)
        while True:
            print ser.readline()
            ser.write('1')
      \end{minted}
      \end{center}  
\end{frame}

\plain{PySerial}{\vspace{-3em}\begin{center}\includegraphics[width=\textwidth, height=3.2in]{plot.png}\end{center}}

\begin{frame}[fragile]
    \frametitle{Flask}
    \begin{center}
        Ufak bir web uygulaması yapmak istiyorum?
    \end{center}  
      \begin{minted}[fontsize=\small]{python}
        from flask import Flask
        app = Flask(__name__)
        
        @app.route("/")
        def hello():
            return "Hello World!"
        
        if __name__ == "__main__":
            app.run()
      \end{minted}
\end{frame}

\plain{Flask}{\vspace{-3em}\begin{center}\includegraphics[width=\textwidth]{pp.jpg}\end{center}}

\begin{frame}[fragile]
    \frametitle{Requests}
    \begin{center}
        API'ler ile kolayca haberleşmek istiyorum?
    \end{center}  
      \begin{minted}[fontsize=\small]{python}
        import requests
        
        link = 'https://api.github.com/user'
        r = requests.get(link, auth=('user', 'pass'))
        r.status_code
        r.text
        r.json()
      \end{minted}
\end{frame}

\begin{frame}[fragile]
    \frametitle{Multiprocessing}
    \begin{center}
        Ders sayfasındaki tüm PDF'leri indirmek istiyorum?
    \end{center}  
      \begin{minted}[fontsize=\small]{python}
        import requests
        from multiprocessing import Pool
        
        links = ["http://a.com/1.pdf", 
                 "http://b.com/2.pdf"]
        
        def saver(link):
            with open(link.split("/")[-1], "wb") as ff:
                ff.write(requests.get(link).content)
        
        pool = Pool(8)
        pool.map(saver, links)
        pool.close()
        pool.join()

      \end{minted}
\end{frame}

\begin{frame}[fragile]
    \frametitle{Pyside}
    \begin{center}
        Masaüstü uygulaması yapmak istiyorum?

    \end{center}  
    \includegraphics[width=\textwidth, height=2.8in]{f.jpg}    
\end{frame}

\begin{frame}[fragile]
    \frametitle{Django}
    \begin{center}
        Kurumsal bir web projesi yapmak istiyorum?
    \end{center}  
\end{frame}

\plain{Pinterest}{\vspace{-3em}\begin{center}\includegraphics[width=\textwidth]{pin.png}\end{center}}

\plain{Instagram}{\vspace{-3em}\begin{center}\includegraphics[width=\textwidth]{ins.png}\end{center}}

\begin{frame}[fragile]
    \frametitle{Socket}
    \begin{center}
        Port tarayıcı yazmak istiyorum?
      
      \begin{minted}[fontsize=\small]{python}
import socket

for port in range(1,1025):  
    sock = socket.socket(socket.AF_INET, socket.SOCK_STREAM)
    result = sock.connect_ex((remoteServerIP, port))
    if result == 0:
        print "Port {}: \t Open".format(port)
    sock.close()
      \end{minted}
      \end{center}
\end{frame}

\begin{frame}[fragile]
    \frametitle{Scrapy}
    \begin{center}
        Bir websitesini, platformu komple indexlemek istiyorum?
    \end{center}  
      \includegraphics[width=\textwidth]{sc.png}
\end{frame}

\begin{frame}[fragile]
    \frametitle{Pygame}
    \begin{center}
        Oyun yapmak istiyorum?
    \end{center}  
      \includegraphics[width=\textwidth,height=2.6in]{pyg.jpg}
\end{frame}

\begin{frame}[fragile]
    \frametitle{Ctypes}
    \begin{center}
        C veya fortran'da yazdıklarımı doğrudan kullanmak istiyorum?
    
      \begin{minted}[fontsize=\small]{python}
      from ctypes import *
      
      libc = CDLL("libc.so.6")
      print libc.time(None)
      
      printf = libc.printf
      printf("Hello, %s\n", "World!")
      \end{minted}
      \end{center}  
\end{frame}

\begin{frame}[fragile]
    \frametitle{Fabric}
    \begin{center}
        Sistem yöneticisiyim, sürekli yapmam gereken bazı işler var?
    
        \begin{minted}[fontsize=\small]{python}
    from fabric.api import *
    
    sudo("mkdir /var/www")
    
    sudo("mkdir /var/www/web-app-one", user="web-admin")
    result = sudo("ls -l /var/www")
    
    run("aptitude    update")
    run("aptitude -y upgrade")
        \end{minted}
        
      \end{center}  
\end{frame}

\begin{frame}[fragile]
    \frametitle{Scipy}
    \begin{center}
        Bilimsel çalışma yapmak istiyorum?
    \end{center}  
      \includegraphics[width=\textwidth,height=2.6in]{bb.png}
\end{frame}

\plain{Scipy}{\vspace{-3em}
    \begin{center}
    \includegraphics[width=\textwidth]{ss.jpg}
    \end{center}
    \begin{center}
    \includegraphics[height=1.2in]{ss1.png}
    \end{center}
    }

\begin{frame}[fragile]
    \frametitle{Pandas}
    \begin{center}
        İstatiksel analiz yapmak istiyorum?
    \end{center}  
      \includegraphics[width=\textwidth,height=2.6in]{pandas.png}
\end{frame}

\plain{Pandas}{\vspace{-3em}\begin{center}\includegraphics[width=\textwidth]{pandas2.png}\end{center}}

\begin{frame}[fragile]
    \frametitle{Opencv}
    \begin{center}
        Görüntü işleme üzerine çalışmak istiyorum?
    \end{center}  
      \includegraphics[width=\textwidth,height=2.6in]{op.jpg}
\end{frame}

\begin{frame}[fragile]
    \frametitle{Simplecv}
    \begin{center}
      \begin{minted}[fontsize=\small]{python}
    from SimpleCV import Image, Color, Display
    
    img = Image('http://i.imgur.com/lfAeZ4n.png')
    
    feats = img.findKeypoints()
    feats.draw(color=Color.RED)
    img.show()
    
    output = img.applyLayers()
    output.save('juniperfeats.png')
      \end{minted}
      \includegraphics[height=1.2in]{jun.png}
      \end{center}  
\end{frame}

\begin{frame}[fragile]
    \frametitle{Exploit}
    \begin{center}
        Exploit geliştirmek istiyorum?
    \end{center}  
      \includegraphics[width=\textwidth,height=2.6in]{dos.png}
\end{frame}

\plain{Peda}{\vspace{-3em}\begin{center}\includegraphics[width=\textwidth]{peda4.png}\end{center}}

\begin{frame}[fragile]
    \frametitle{Cmd}
    \begin{center}
         Interaktif komut satırı uygulamaları yapmak istiyorum?
    \end{center}  
      \includegraphics[width=\textwidth,height=2.6in]{sc1.png}
\end{frame}

\begin{frame}[fragile]
    \frametitle{PIL}
    \begin{center}
         QR bahçesini nasıl biçebilirim?
    \end{center}  
      \includegraphics[width=\textwidth,height=2.6in]{qr.png}
\end{frame}

\begin{frame}[fragile]
    \frametitle{QR}
    \begin{center}
      \begin{minted}[fontsize=\small]{python}
import Image
import sys
from qr import *

img = Image.open("qrgarden.png")
img = img.convert('RGB')
for y in range(0, 100):
    for x in range(100):
        qrtxt = scan(img, 29*module_size*x, 29*module_size*y)
        print read_qr(qrtxt)
      \end{minted}
      \end{center}  
\end{frame}

\begin{frame}[fragile]
    \frametitle{Airprobe}
    \begin{center}
         Telefon dinleme? :)
    \end{center}  
      \includegraphics[width=\textwidth,height=2.6in]{airprobe.jpg}
\end{frame}

\begin{frame}[fragile]
    \frametitle{Nasıl öğrenebilirim?}
    \begin{itemize}
        \item Learn Python the hard way
        \item The Hitchhiker’s guide to Python
        \item Google’s Python class
        \item Python tracks at Codecademy
        \item PySchools
        \item Python Monk
        \item Dive into Python
        \item Think Python
        \item Invent with Python
        \item A byte of Python\ldots
    \end{itemize}  
\end{frame}

\begin{frame}[fragile]
    \frametitle{Nasıl takip ederim?}
    \begin{itemize}
        \item Python subreddit
        \item Python Weekly
        \item Pycoder’s
        \item Python planet
        \item Pycon
        \item PyIstanbul
        \item Github\ldots
    \end{itemize}  
\end{frame}

\begin{frame}{Sonuc}

\begin{center}Sunum slaytlarına aşağıdaki linkten erişebilirsiniz

\url{github.com/halitalptekin/karabuk-slides}\end{center}


\end{frame}

\plain{}{Sorular?}

\end{document}
